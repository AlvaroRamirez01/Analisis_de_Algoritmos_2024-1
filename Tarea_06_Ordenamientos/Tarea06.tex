\documentclass[answers, 10pt]{exam}

\usepackage[spanish]{babel}
\usepackage[shortlabels]{enumitem}
\usepackage[T1]{fontenc}
\usepackage{graphicx}
\usepackage[cache=false,outputdir=./build]{minted}
\usepackage{lmodern}
\usepackage{wrapfig}
\usepackage{xcolor, color}
\definecolor{LightGray}{gray}{0.9}

\newcommand{\materia}{Analisis de Algoritmos 2024-1}
\newcommand{\tarea}{Tarea 06: Ordenamientos}
\newcommand{\fecha}{\today}
\newcommand{\profesor}{Profesor(a): María de Luz Gasca Soto}
\newcommand{\ayudantes}{
  Rodrigo Fernando Velázquez Cruz \\
  Teresa Becerril Torres
}
\newcommand{\alumnos}{
  Alvaro Ramirez Lopez \textbf{N° cuenta:} 316276355
}

\decimalpoint{}
\graphicspath{{Imagenes}}
% \colorsolutionboxes
\shadedsolutions

% \definecolor{SolutionBoxColor}{rgb}{0,128,255}
% \definecolor{SolutionColor}{rgb}{0,204,255}
\definecolor{SolutionColor}{rgb}{0,128,255}

\renewcommand{\familydefault}{\sfdefault}

\extrawidth{1.56cm}
\extraheadheight[1.5in]{-0.25in}
\extrafootheight[-0.175in]{-0.375in}
\firstpageheader{
}{
  \begin{minipage}[c]{3.5cm}
    \includegraphics[width=3.5cm]{fc.png}
  \end{minipage}
  \begin{minipage}[c]{11.0cm}
    {\bfseries\huge\materia{} \\[2mm]
      \LARGE \tarea{} \\
      \large Profesor:} \profesor{} \\
    \hspace{0.1cm}
    {\bfseries\large Ayudantes:}
    \begin{minipage}[t]{8.5cm}
      \ayudantes{}\vspace{0.1cm}
    \end{minipage}\hfill\break{}
    {\bfseries\large Alumno:}
    \begin{minipage}[t]{8.5cm}
      \alumnos{}\\
    \end{minipage}\hfill\break{}
  \end{minipage}
  \begin{minipage}[c]{3.25cm}
    \includegraphics[width=3.25cm]{unam.png}
  \end{minipage}
}{}
\runningheader{\materia}{\tarea}{\fecha}
\runningheadrule{}
\footer{}{Página \thepage\ de \numpages}{}
\footrule{}
\renewcommand{\solutiontitle}{\noindent\textbf{Solución:}\par\noindent}



\begin{document}
\begin{questions}

  \question{El Problema de Selección consiste en encontrar el $k$-ésimo elemento más pequeño de un conjunto de $n$ datos, $k \leq n$.

    Considere los algoritmos de ordenamiento:
    \begin{enumerate}[(a)]
      \item Merge Sort
      \item Selection Sort
      \item Quick Sort
      \item Heap Sort
      \item Insertion Sort
      \item Shell Sort
      \item Local Insertion Sort
    \end{enumerate}

    Pregunta: ¿Cuáles de las estrategias usadas por los algoritmos anteriores, nos ayudan a resolver el Problema de Selección, sin tener que ordenar toda la secuencia? Suponga $k$ tal que $1 < k < n$.

    Justifique, para cada inciso, sus respuestas.
  }
  \begin{solution}
    \begin{enumerate}
      \item \textbf{Merge Sort:} Este algoritmo puede modificarse para resolver el
            Problema de Selección. En lugar de hacer merge completo de las sublistas,
            podemos detener el proceso cuando queden k elementos sin mezclar en
            el conjunto grande, es decir que haremos recursion hasta el nivel k donde
            el tamaño de los subconjuntos sea 1 o 2, y luego haremos un merge pero solo
            con los elementos menores de esos subconjuntos, es decir que nos quedaria un subconjunto
            de $n \verb| div | 2$ elementos, ahi las comparaciones serian a lo mas $log (n)$.

      \item \textbf{Selection Sort:} No ayuda a resolver el Problema de Selección sin
            ordenar toda la secuencia, ya que se basa en intercambiar el elemento
            mínimo sucesivamente hasta ponerlo en la posición correcta.

      \item \textbf{Quick Sort:} Al igual que Merge Sort, puede modificarse para
            resolver el Problema de Selección. Podemos detener el proceso cuando el
            pivote ocupe la posición k, retornando ese elemento. Es decir que solo
            haremos uso de un lado de la particion, asi descartamos los elementos mayores
            al pivote y seguimos haciendo recursion hasta que el tamaño del subconjunto sea 1 o 2.

      \item \textbf{Heap Sort:} Puede usarse el heap mínimo (min heap) para
            extraer los k elementos más pequeños en orden en O(k log n) tiempo. Esto
            resuelve el Problema de Selección sin ordenar toda la secuencia.

      \item \textbf{Insertion Sort:} No ayuda a resolver el Problema de Selección
            sin ordenar toda la secuencia, ya que ordena insertando elementos uno a uno
            en posiciones anteriores.

      \item \textbf{Shell Sort:} Tampoco ayuda sin ordenar toda la secuencia, ya
            que es una variante de Insertion Sort.

      \item \textbf{Local Insertion Sort:} Al igual que Insertion Sort, debe
            ordenar toda la secuencia, ademas que la referencia que se guarda sobre el
            ultimo elemento insertado no nos asegura que sea el elemento k-esimo mas
            pequeño.
    \end{enumerate}
  \end{solution}
  \question{Resolver el Problema de Selección usando el proceso Partition. Además indique el desempeño computacional de la estrategia.}
  \begin{solution}
    Aquí va la solución.
  \end{solution}
  \question{Desarrollar uno de los siguientes ejercicios:
    \begin{enumerate}
      \item[(A)] Desarrollar un algoritmo que genere listas de datos que resulten
        ser el peor caso para el Shell Sort para las secuencias de Shell y para las
        secuencias de Hibbart.
      \item[(B)] Desarrolle un algoritmo que genere listas de datos que resulten
        ser el peor caso para el Quick Sort cuando el pivote es $S[n$ \verb|div| $2]$
    \end{enumerate}}
  \begin{solution}
    \begin{enumerate}[(A)]
      \item \textbf{Desarrollo del algoritmo que genera el peor caso para el Shell Sort para las secuencias de Shell}

            Para generar el peor caso para las secuencias de Shell debemos diseñar una
            lista de datos en orden inverso de modo que shell sort tenga que hacer
            muchas comparaciones e intercambios en cada paso. A continuación se muestra
            el algoritmo que genera el peor caso para el Shell Sort para las secuencias
            de Shell.

            \begin{minted}[
       frame=lines,
       framesep=2mm,
       baselinestretch=1.2,
       bgcolor=LightGray,
       fontsize=\footnotesize,
       linenos
       ]{Python}
   import sys
   
   def worst_case_shell_sort_shell(seq):
     n = len(seq)
     h = 1
     while h < n // 3:
       h = 3 * h + 1
    
     worst_case_list = list(range(n, 0, -1))
     return worst_case_list
    
   # Ejemplo de uso
   n = int(sys.argv[1])
   worst_case_list = worst_case_shell_sort_shell(list(range(1, n + 1)))
   print("Peor caso para Shell Sort con secuencias de Shell (n = {}):".format(n))
   print(worst_case_list)
 \end{minted}

            \textbf{Desarrollo del algoritmo que genera el peor caso para Shell Sort para las secuencias de Hibbart}

            Para generar el peor caso para las secuencias de Hibbart debemos diseñar una
            lista de datos que sea la concatenación de varias secuencias ordenadas de manera inversa.

            \begin{minted}[frame=lines,
       framesep=2mm,
       baselinestretch=1.2,
       bgcolor=LightGray,
       fontsize=\footnotesize,
       linenos
       ]{Python}
   import sys

   def worst_case_shell_sort_hibbard(seq):
     n = len(seq)
     gaps = [2**k - 1 for k in range(int(math.log2(n)) + 1)]
     worst_case_list = []

   for gap in gaps:
       sublist = list(range(n, 0, -1))[:gap]
       worst_case_list += sublist

   return worst_case_list

   # Ejemplo de uso
   n = int(sys.argv[1])
   worst_case_list = worst_case_shell_sort_hibbard(list(range(1, n + 1)))
   print("Peor caso para Shell Sort con secuencias de Hibbard (n = {}):".format(n))
   print(worst_case_list)
     \end{minted}

      \item \textbf{Desarrollo del algoritmo que genera el peor caso para el Quick Sort}

            Para generar el peor caso para el algoritmo de Quick Sort cuando el pivote
            se elige como el elemento en la posición S[n div 2], debemos diseñar una
            lista de datos que esté ordenada de manera descendente o ascendente, el siguiente
            algoritmo crea una lista ordenada de manera descendente y luego intercambia
            el elemento en la posición n div 2 (la mitad de la lista) con el primer
            elemento. Esto garantiza que el pivote siempre divida la lista en dos
            sublistas desiguales, lo que resulta en el peor caso de rendimiento para
            Quick Sort con el pivote en S[n div 2].

            \begin{minted}[
        frame=lines,
        framesep=2mm,
        baselinestretch=1.2,
        bgcolor=LightGray,
        fontsize=\footnotesize,
        linenos
        ]{Python}
  import sys

  def worst_case_quick_sort(n):
    if n <= 0:
      return []

    # Crear una lista ordenada de manera descendente
    worst_case_list = list(range(n, 0, -1))
    
    # Intercambiar el elemento en la posición n div 2 con el primer elemento
    worst_case_list[0], worst_case_list[n // 2] = worst_case_list[n // 2], worst_case_list[0]
    
    return worst_case_list

  # Ejemplo de uso
  n = int(sys.argv[1])
  worst_case_list = worst_case_quick_sort(n)
  print("Peor caso para Quick Sort (n = {}):".format(n))
  print(worst_case_list)

      \end{minted}
    \end{enumerate}

  \end{solution}
  \question{\textbf{Problema Opcional} $\Phi$: Suponga que tiene $n$ intervalos cerrados sobre la línea real: $[a(i), b(i)]$, con $1 \leq i \leq n$. Encontrar la máxima $k$ tal que existe un punto $x$ que es cubierto por los $k$ intervalos.
    \begin{enumerate}[(a)]
      \item Proporcione un algoritmo que solucione el problema $\Phi$.
      \item Justifique que su propuesta de algoritmo es correcta.
      \item Calcule, con detalle, la complejidad computacional de su propuesta.
      \item Proporcione un pseudo-código del algoritmo propuesto.
    \end{enumerate}}
  % \begin{solution}
  % 	\begin{enumerate}[(a)]
  % 		\item \textbf{Proporcione un algoritmo que solucione el problema $\Phi$.}
  % 		      \begin{enumerate}[1.]
  % 			      \item Combine los extremos de los intervalos en un único conjunto 
  %             de puntos, junto con una etiqueta que indique si el punto es un 
  %             extremo izquierdo (L) o derecho (R) del intervalo.

  % 			      \item Ordene este conjunto de puntos en orden ascendente.

  % 			      \item Inicializa una variable k en 0 y un contador maxK en 0.

  % 			      \item Recorre los puntos ordenados y, para cada punto, aumenta k 
  %             si es un extremo izquierdo y disminuye k si es un extremo derecho. 
  %             Mantén un seguimiento de maxK como el máximo valor alcanzado por k.

  % 			      \item Cuando completes el recorrido de todos los puntos, maxK 
  %             contendrá la respuesta al problema.

  % 		      \end{enumerate}
  % 		\item \textbf{Justifique que su propuesta de algoritmo es correcta.}

  % 		      Este algoritmo es correcto porque explora todos los puntos de los 
  %           intervalos en orden ascendente. Cuando se encuentra un extremo 
  %           izquierdo, incrementa k, lo que significa que se ha cubierto un 
  %           nuevo intervalo. Cuando se encuentra un extremo derecho, disminuye k, 
  %           lo que indica que un intervalo ha terminado de cubrirse. En todo 
  %           momento, maxK registra el valor máximo de k observado, que es la 
  %           máxima cantidad de intervalos que cubren un punto.

  % 		\item \textbf{Calcule, con detalle, la complejidad computacional de su propuesta}

  % 		      La complejidad computacional de este algoritmo es dominada por la 
  %           etapa de ordenamiento de los puntos, que toma O(n log n) tiempo 
  %           utilizando un algoritmo de ordenamiento eficiente, como QuickSort o 
  %           MergeSort. El bucle que recorre los puntos toma tiempo lineal O(n), 
  %           ya que cada punto se procesa una vez. Por lo tanto, la complejidad 
  %           total es O(n log n).

  % 		\item \textbf{Proporcione un pseudo-código del algoritmo propuesto.}
  % 		      \begin{minted}[
  %       frame=lines,
  %       framesep=2mm,
  %       baselinestretch=1.2,
  %       bgcolor=LightGray,
  %       fontsize=\footnotesize,
  %       linenos
  %       ]{Python}
  %   def max_coverage(intervals):
  %     points = []
  %     for interval in intervals:
  %       points.append((interval[0], 'L'))  # Extremo izquierdo
  %       points.append((interval[1], 'R'))  # Extremo derecho

  %   points.sort()  # Ordenar los puntos

  %   k = 0
  %   maxK = 0

  %   for point in points:
  %       if point[1] == 'L':
  %           k += 1
  %       else:
  %           k -= 1
  %       maxK = max(maxK, k)

  %   return maxK

  %     \end{minted}
  % 	\end{enumerate}
  % \end{solution}

  % \begin{solutionbox}{2in}
  %   Once upon a midnight dreary, while I pondered, weak and weary, Over
  %   many a quaint and curious volume of forgotten lore--- While I
  %   nodded, nearly napping, suddenly there came a tapping, As of some
  %   one gently rapping, rapping at my chamber door. ‘‘\,’Tis some
  %   visitor,’’ I muttered, ‘‘tapping at my chamber door--- Only this and
  %   nothing more.’’
  %   \end{solutionbox}
\end{questions}
\end{document}

