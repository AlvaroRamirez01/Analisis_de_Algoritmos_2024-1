\documentclass[answers]{exam}

\usepackage[spanish]{babel}
\usepackage[shortlabels]{enumitem}
\usepackage[T1]{fontenc}
\usepackage{graphicx}
\usepackage{lmodern}
\usepackage{wrapfig}
\usepackage{xcolor, color}

\newcommand{\materia}{Analisis de Algoritmos 2024-1}
\newcommand{\tarea}{Tarea 06: Ordenamientos}
\newcommand{\fecha}{\today}
\newcommand{\profesor}{Profesor(a): María de Luz Gasca Soto}
\newcommand{\ayudantes}{
  Rodrigo Fernando Velázquez Cruz \\
  Teresa Becerril Torres
}
\newcommand{\alumnos}{
  Alvaro Ramirez Lopez \textbf{N° cuenta:} 316276355
}

\decimalpoint{}
\graphicspath{{Imagenes}}
\colorsolutionboxes
\shadedsolutions

% \definecolor{SolutionBoxColor}{rgb}{0,128,255}
% \definecolor{SolutionColor}{rgb}{0,204,255}
\definecolor{SolutionColor}{rgb}{0,128,255}

\renewcommand{\familydefault}{\sfdefault}

\extrawidth{1.56cm}
\extraheadheight[1.5in]{-0.25in}
\extrafootheight[-0.175in]{-0.375in}
\firstpageheader{
}{
  \begin{minipage}[c]{3.5cm}
    \includegraphics[width=3.5cm]{fc.png}
  \end{minipage}
  \begin{minipage}[c]{11.0cm}
    {\bfseries\huge\materia{} \\[2mm]
      \LARGE \tarea{} \\
      \large Profesor:} \profesor{} \\
    \hspace{0.1cm}
    {\bfseries\large Ayudantes:}
    \begin{minipage}[t]{8.5cm}
      \ayudantes{}
    \end{minipage}\hfill\break{}
    {\bfseries\small Alumnos:}
    \begin{minipage}[t]{8.5cm}
      \alumnos{}
    \end{minipage}\hfill\break{}
  \end{minipage}
  \begin{minipage}[c]{3.25cm}
    \includegraphics[width=3.25cm]{unam.png}
  \end{minipage}
}{}
\runningheader{\materia}{\tarea}{\fecha}
\runningheadrule{}
\footer{}{Página \thepage\ de \numpages}{}
\footrule{}
\renewcommand{\solutiontitle}{\noindent\textbf{Solución:}\par\noindent}



\begin{document}
\begin{questions}

  \question{El Problema de Selección consiste en encontrar el $k$-ésimo elemento más pequeño de un conjunto de $n$ datos, $k \leq n$.

  Considere los algoritmos de ordenamiento:
  \begin{enumerate}[(a)]
  \item Merge Sort
  \item Selection Sort
  \item Quick Sort
  \item Heap Sort
  \item Insertion Sort
  \item Shell Sort
  \item Local Insertion Sort
  \end{enumerate}
  
  Pregunta: ¿Cuáles de las estrategias usadas por los algoritmos anteriores, nos ayudan a resolver el Problema de Selección, sin tener que ordenar toda la secuencia? Suponga $k$ tal que $1 < k < n$.
  
  Justifique, para cada inciso, sus respuestas.
  }
  \begin{solution}
    Aquí va la solución
  \end{solution}
  \question{Resolver el Problema de Selección usando el proceso Partition. Además indique el desempeño computacional de la estrategia.}
  \begin{solution}
    Aquí va la solución.
  \end{solution}
  \question{Desarrollar uno de los siguientes ejercicios:
    \begin{enumerate}
      \item[(A)] Desarrollar un algoritmo que genere listas de datos que resulten ser el peor caso para el Shell Sort para las secuencias de Shell y para las secuencias de Hibbart.
      \item[(B)] Desarrolle un algoritmo que genere listas de datos que resulten ser el peor caso para el Quick Sort cuando el pivote es $S[n$ \verb|div| $2]$
    \end{enumerate}}
  \begin{solution}
    Aquí va la solución.
  \end{solution}
  \question{\textbf{Problema Opcional} $\Phi$: Suponga que tiene $n$ intervalos cerrados sobre la línea real: $[a(i), b(i)]$, con $1 \leq i \leq n$. Encontrar la máxima $k$ tal que existe un punto $x$ que es cubierto por los $k$ intervalos.
  \begin{enumerate}
  \item Proporcione un algoritmo que solucione el problema $\Phi$.
  \item Justifique que su propuesta de algoritmo es correcta.
  \item Calcule, con detalle, la complejidad computacional de su propuesta
  \item Proporcione un pseudo-código del algoritmo propuesto.
  \end{enumerate}}
  \begin{solution}
    Aquí va la solución.
  \end{solution}

  \begin{solutionbox}{2in}
    Once upon a midnight dreary, while I pondered, weak and weary, Over
    many a quaint and curious volume of forgotten lore--- While I
    nodded, nearly napping, suddenly there came a tapping, As of some
    one gently rapping, rapping at my chamber door. ‘‘\,’Tis some
    visitor,’’ I muttered, ‘‘tapping at my chamber door--- Only this and
    nothing more.’’
    \end{solutionbox}
\end{questions}
\end{document}

