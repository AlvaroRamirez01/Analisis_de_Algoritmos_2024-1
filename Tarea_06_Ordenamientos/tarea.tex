\documentclass{article}
\title{Análisis de Algoritmos\\Tarea: Ordenamientos}
\author{Profra: Luz Gasca Soto\\Ayudantes: Teresa Becerril & Rodrigo F. Velázquez}
\date{13 de Octubre de 2023}

\begin{document}
\maketitle

\section{El Problema de Selección}

El Problema de Selección consiste en encontrar el $k$-ésimo elemento más pequeño de un conjunto de $n$ datos, $k \leq n$.

Considere los algoritmos de ordenamiento:
\begin{enumerate}
\item Merge Sort
\item Selection Sort
\item Quick Sort
\item Heap Sort
\item Insertion Sort
\item Shell Sort
\item Local Insertion Sort
\end{enumerate}

Pregunta: ¿Cuáles de las estrategias usadas por los algoritmos anteriores, nos ayudan a resolver el Problema de Selección, sin tener que ordenar toda la secuencia? Suponga $k$ tal que $1 < k < n$.

Justifique, para cada inciso, sus respuestas.

\section{Problema de Selección usando Partition}
Resolver el Problema de Selección usando el proceso Partition. Además indique el desempeño computacional de la estrategia.

\section{Ejercicios}
Desarrollar uno de los siguientes ejercicios:
\begin{enumerate}
\item[(A)] Desarrollar un algoritmo que genere listas de datos que resulten ser el peor caso para el Shell Sort para las secuencias de Shell y para las secuencias de Hibbart.

\item[(B)] Desarrolle un algoritmo que genere listas de datos que resulten ser el peor caso para el Quick Sort cuando el pivote es $S[n$ \verb|div| $2]$
\end{enumerate}

\textbf{O. Problema} $\Phi$: Suponga que tiene $n$ intervalos cerrados sobre la línea real: $[a(i), b(i)]$, con $1 \leq i \leq n$. Encontrar la máxima $k$ tal que existe un punto $x$ que es cubierto por los $k$ intervalos.
\begin{enumerate}
\item Proporcione un algoritmo que solucione el problema $\Phi$.
\item Justifique que su propuesta de algoritmo es correcta.
\item Calcule, con detalle, la complejidad computacional de su propuesta
\item Proporcione un pseudo-código del algoritmo propuesto.
\end{enumerate}

\section{Fecha de Entrega y Examen}
Martes 23 de Octubre del 2023

¡Suerte!

\end{document}